Meta data unit have a particular format.
The first bytes of the file contain \emph{RS-META}\footnote{This is a 0 string terminated. So there are 8 characters reserved.}.
Then start real contain.
See Figure~\ref{fig:META-FILE-FORMAT} for more detail.
The next 8 bytes are the same than 8 first.
\begin{itemize}
 \item If the bytes in range [8..15] are not the same than in range [0..7] that means file is encrypted.
The key is not store by Share Resource library.
User software must delegate a key and the library will search the meta units that unlock.
The key is a symetric key.
 \item If the bytes in range [8..15] are the same than in range [0..7] that means unit is not encrypted.
\end{itemize}
\begin{figure}[htbp]
  \centering
\begin{bytefield}[bitwidth=2em]{16}
    \bitheader{0,7-8,15} \\

    \begin{rightwordgroup}{file header}
      % We have to do the \parbox explicitly in the next line because
      % \hyperlink typesets its argument in horizontal mode.\parbox{\width}{}
      \bitbox{1}{R} &
      \bitbox{1}{S} &
      \bitbox{1}{-} &
      \bitbox{1}{M} &
      \bitbox{1}{E} &
      \bitbox{1}{T} &
      \bitbox{1}{A} &
      \bitbox{1}{\textbackslash 0} &
      \bitbox{1}{R} &
      \bitbox{1}{S} &
      \bitbox{1}{-} &
      \bitbox{1}{M} &
      \bitbox{1}{E} &
      \bitbox{1}{T} &
      \bitbox{1}{A} &
      \bitbox{1}{\textbackslash 0}
   \end{rightwordgroup} \\

  \begin{rightwordgroup}{Meta data}
    \wordbox{2}{First meta field object} \\
    \wordbox[lrt]{1}{%
      \parbox{0.6\width}{\centering (Meta field objects)}} \\
    \skippedwords \\
    \wordbox{2}{Last meta field object}
  \end{rightwordgroup}
  \end{bytefield}
  \caption{META file format.}
  \label{fig:META-FILE-FORMAT}
  \end{figure}




Data in the \emph{META unit} are serialized\footnote{That means that data can be stored with no order. As the \hyperlink{fields:header-index}{header index} is not to be sure to be in the first place.}.

Each field have the same scheme (Figure~\ref{fig:META-FIELD-FORMAT}):

\begin{figure}[htbp]
  \centering
\begin{bytefield}{32}
    \bitheader{0,7-8,15-16,23-24,31} \\

    \begin{rightwordgroup}{header}
      % We have to do the \parbox explicitly in the next line because
      % \hyperlink typesets its argument in horizontal mode.\parbox{\width}{}
      \wordbox{1}{\hyperlink{META-FIELD-Size}{\centering Size}} \\
      \wordbox{1}{\hyperlink{META-FIELD-Type}{type}}
    \end{rightwordgroup} \\

  \begin{rightwordgroup}{Object}
    \wordbox[lrt]{1}{%
      \parbox{0.6\width}{\centering (Object contain.)}} \\
    \skippedwords \\
    \wordbox[lrb]{1}{}
  \end{rightwordgroup}
  \end{bytefield}
  \caption{META object field.}
  \label{fig:META-FIELD-FORMAT}
  \end{figure}


  \begin{itemize}
   \item \hypertarget{META-FIELD-Size}{Size} of the object in byte. This include header, for a nul size object the size is 8.
   \item \hypertarget{META-FIELD-Type}{Type} of the object. Can be \emph{ResourceId} for example.
          All types are listed at the \hyperlink{META-FIELD-All-Types}{here}.
  \end{itemize}





The list of fields are in table \ref{tab:field-list}.

\begin{table}[htbp]
  \begin{tabular}{l|c|c|c}
    Name & introduce version & obsolete version & last version \\
    \hline
    \hyperlink{fields:header-index}{Header index} & 0.1 & & \\
    \hyperlink{fields:resource-id}{Resource Id} & 0.1 & & \\
    \hyperlink{fields:data-index}{Data index} & 0.1 & & \\
    \hyperlink{fields:data-file}{Data file} & 0.1 & & \\
    \hyperlink{fields:auth-user-index}{Authorized user index} & 0.1 & & \\
    \hyperlink{fields:auth-user}{Authorized user} & 0.1 & & \\
    \hyperlink{fields:version}{Resource version} & 0.1 & & \\
    \hyperlink{fields:parents}{Parents} & 0.1 & & \\
  \end{tabular}
  \caption{List of fields in Meta File}
  \label{tab:field-list}
\end{table}

\hypertarget{fields:header-index}{\subsubsection{Header index}}
\field{Header index}{Optional}{Header index is used to help to get information faster than reading fields one by one}



\hypertarget{fields:resource-id}{\subsubsection{Resource Id}}

\hypertarget{fields:data-index}{\subsubsection{Data index}}

\hypertarget{fields:data-file}{\subsubsection{Data File}}

\hypertarget{fields:auth-user-index}{\subsubsection{Authorized user index}}

\hypertarget{fields:auth-user}{\subsubsection{Authorized user}}

\hypertarget{fields:version}{\subsubsection{Resource version}}

\hypertarget{fields:parents}{\subsubsection{Parents}}





