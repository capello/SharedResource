The data are store in data unit. The format of these units is composed with a header, data and a footer.
Detail are in figure~\ref{fig:DATA-FILE-FORMAT}~:
\begin{figure}[htbp]
  \centering
\begin{bytefield}[bitwidth=2em]{16}
    \bitheader{0,7-8,15} \\

    \begin{rightwordgroup}{file header}
      % We have to do the \parbox explicitly in the next line because
      % \hyperlink typesets its argument in horizontal mode.\parbox{\width}{}
      \wordbox{2}{Resource ID}
   \end{rightwordgroup} \\

  \begin{rightwordgroup}{Data\\\emph{encrypted}}
    \wordbox{5}{%
      \parbox{0.6\width}{\centering (DATA)}}
  \end{rightwordgroup} \\
  \begin{rightwordgroup}{Footer\\\emph{encrypted}}
    \wordbox{2}{SHA1 of DATA}
  \end{rightwordgroup} \\

  \end{bytefield}
  \caption{DATA file format.}
  \label{fig:DATA-FILE-FORMAT}
\end{figure}

The first field is the resource id. With this field, we can retreive the META files of resource.
Data part is data array sent by software. The SHA1 is the SHA1 of data. The parts DATA and footer can be encrypted depending of META files.
