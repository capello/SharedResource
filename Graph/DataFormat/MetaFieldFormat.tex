Each field have the same scheme (Figure~\ref{fig:META-FIELD-FORMAT}):

\begin{figure}[htbp]
  \centering
  \begin{bytefield}{32}
    \bitheader{0,7-8,15-16,23-24,31} \\

    \begin{rightwordgroup}{header}
      % We have to do the \parbox explicitly in the next line because
      % \hyperlink typesets its argument in horizontal mode.\parbox{\width}{}
      \wordbox{1}{\hyperlink{META-FIELD-Size}{\centering Size}} \\
      \wordbox{1}{\hyperlink{META-FIELD-Type}{type}}
    \end{rightwordgroup} \\

    \begin{rightwordgroup}{Object}
      \wordbox[lrt]{1}{%
        \parbox{0.6\width}{\centering (Object contain.)}} \\
      \skippedwords \\
      \wordbox[lrb]{1}{}
    \end{rightwordgroup}
  \end{bytefield}
  \caption{META object field.}
  \label{fig:META-FIELD-FORMAT}
\end{figure}


\begin{itemize}
  \item \hypertarget{META-FIELD-Size}{Size} of the object in byte. This include header, for a nul size object the size is 8.
  \item \hypertarget{META-FIELD-Type}{Type} of the object. Can be \emph{ResourceId} for example.
        All types are listed at the \hyperlink{META-FIELD-All-Types}{here}.
\end{itemize}



